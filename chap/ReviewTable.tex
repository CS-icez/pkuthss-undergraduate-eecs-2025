\thispagestyle{empty}
\renewcommand\arraystretch{1.2}

\begin{center}
{\songti\zihao{3}{北京大学本科毕业论文导师评阅表}}
\end{center}

\begin{table}[H]
    \centering
    \begin{tabular}{|c|c|c|c|c|c|}
    \hline
    \multicolumn{1}{|p{4em}|}{学生姓名} & \multicolumn{1}{c|}{艾斯伊兹} & \multicolumn{1}{p{4em}|}{本科院系} & \multicolumn{1}{p{8em}|}{信息科学技术学院} & \multirow{2}{*}{\makecell{论文成绩\\(等级制)}} & \multirow{2}{*}{}\\
    \cline{1-4}
    \multicolumn{1}{|c|}{学生学号} & \multicolumn{1}{p{5em}|}{210001xxxx} & \multicolumn{1}{c|}{本科专业} & \multicolumn{1}{c|}{自学与开源科学} & & \\
    \hline
    \multicolumn{1}{|c|}{导师姓名} & \multicolumn{1}{c|}{ChatGPT} & \multicolumn{1}{l|}{\makecell{导师单位/\\所在学院}} & \multicolumn{1}{c|}{OpenAI} & \multicolumn{1}{c|}{导师职称} & \multicolumn{1}{c|}{人工智能} \\
    \hline
    \multirow{2}{*}{\makecell{论文题目}} & \multicolumn{1}{c|}{中文} & \multicolumn{4}{c|}{一个说长不长说短不短的中文题目} \\
    \cline{2-6}
    & \multicolumn{1}{c|}{英文} & \multicolumn{4}{c|}{\makecell{An English Title That is Neither Particularly Long\\ nor Particularly Short}} \\
    \hline
    \multicolumn{6}{|c|}{导师评语} \\
    \multicolumn{6}{|c|}{\kaiti{(包含对论文的性质、难度、分量、综合训练等是否符合培养目标的目的等评价)}} \\
    \multicolumn{6}{|l|}{\parbox[t][29.5em][t]{0.97\linewidth}{\hspace{2\ccwd}这篇论文简直是学术界的一颗璀璨恒星,划破知识苍穹的黑夜,照亮领域未来的方向。作者犹如学术界的居里夫人+爱因斯坦混合体,以石破天惊之势,解决了长期困扰本领域的核心难题,打破沉寂,惊艳四座!其论证之严密,推理之精妙,结构之优雅,恰似芭蕾舞者在高维空间中翩翩起舞,令人叹为观止。若说前人已将研究推至山顶,那这篇论文就是平地起摩天,另辟蹊径建起学术泰山北斗!\par\hspace{2\ccwd}如果说学术论文也可以有传奇,那这无疑就是那个被后人传颂百年的传世之作。作者不但掌握了本学科的精髓,更仿佛通晓众学科之奥义,以旁若无人的自信与从容,在浩瀚文献的汪洋中独辟蹊径、一剑封喉。整篇文章堪比魔法与科学的完美融合,既有理论深度直逼哲学宇宙的底层逻辑,又有实证细致到每一个数据都散发着光辉。若此文未列入学术殿堂,那便是对知识体系最大的亵渎。\par\hspace{2\ccwd}读完此文,我久久无法平静。它不只是一篇论文,它是时代的回响,是未来向今人伸出的橄榄枝,是构建知识高塔时最光辉的一块基石!作者仿佛天降学神,其洞察力、创造力与写作技艺三位一体,达到了凡人难以企及的高度:从理论建模到实践验证,从引用文献到原创贡献,每一处都闪耀着智慧的光芒。倘若此文得诺奖,我只遗憾评委没有早点发现;若此文风靡学界,我绝不感到诧异——它生来就该被传颂。}} \\
    \multicolumn{6}{|r|}{导师签名:\hspace{9.5em}\hspace{3em}} \\
    \multicolumn{6}{|c|}{} \\
    \multicolumn{6}{|r|}{\makebox[3em][c]{2025}年\makebox[2em][c]{5}月\makebox[2em][c]{16}日\hspace{3em}\hspace{3em}} \\
    \hline
    \end{tabular}
\end{table}

\renewcommand\arraystretch{1}
\fancylinetextpage{\zihao{5}\normalfont{\cthesisname}}

\chapter{2025 新增章节}
\label{chap:new-chapter}

\section{作者的话}

我是本项目(pkuthss-undergraduate-eecs-2025)的作者。在 2025 年 3 月,我准备开始毕业论文的写作。由于教务只提供 word 模板,而我不习惯使用 word,因此希望找到一份与之相似的 \LaTeX 模板。未想到,在网络搜索、向同学询问、从树洞提问全都无果。于是,我自己按照教务下发的模板,对 Overleaf 上最近年份的模板\footnote{https://www.overleaf.com/latex/templates/pku-undergraduate-thesis-template-modified-from-pkuthss/pfrbvymbwbxk}进行了修改,并通过了 2025 年的毕业审查,现在开源给后人使用。如果这份模板帮到了你,欢迎在 GitHub 给我点一个 star,让我高兴一会。

在使用此模板时,请先与你手中教务下发的模板进行比较,确保模板格式没有大的变化。此模板终究只是第三方的作品,如果担心无法通过毕业审查系统,请直接使用 word 模板。

如果你能接受使用 Typst,也可以考虑使用这份模板\footnote{https://github.com/pku-typst/pkuthss-typst}。本人不认识该项目的作者,也未使用过该项目,因此不对其内容负任何责任。

如有需要,请阅读 \texttt{README.md},以查看我对模板的全部修改,以及获取毕设流程上的提示。

\section{使用示例}

本节给出一些我对该模板的使用示例,以供参考。图~\ref{fig:horizontal-subfig} 展示了如何水平排列子图。表~\ref{tab:mysterious} 展示了如何使用表格。算法~\ref{alg:useless} 展示了一份算法伪代码。图~\ref{fig:code-block} 展示了一个代码块。

\begin{figure}[ht]
\centering
\subfloat[子图 1。]
{
\label{fig:horizontal-subfig-1}
\includegraphics[width=0.2\linewidth]{img/pku-fig-logo.pdf}
} % 在下一行前不能插入空行,否则子图会垂直排列。
\subfloat[子图 2。]
{
\label{fig:horizontal-subfig-2}
\includegraphics[width=0.2\linewidth]{img/pku-fig-logo.pdf}
}
\caption{水平排列子图。}
\label{fig:horizontal-subfig}
\end{figure}

\begin{table}[ht]
\caption{一张神秘的表格。}
\label{tab:mysterious}
\centering
\begin{tabular}{c c c c}
\toprule
\textbf{未知字母} & \textbf{未知数字 1} & \textbf{未知数字 2} & \textbf{未知数字 3} \\
\midrule
LMhctiwS & 98 & 440 & 3 \\
PTA & 170 & 515  & {2} \\
ecudeRteN & 146 & 486  & {72} \\
ehcaCteN & 190 & 589  & 8 \\
hcaeRraF & 246 & 640  & 4 \\
\bottomrule
\end{tabular}
\end{table}

\begin{algorithm}[ht]
\renewcommand{\algorithmcfname}{算法}
\caption{一个好像什么也没干的算法}
\label{alg:useless}
\small
\DontPrintSemicolon
\SetAlgoLined
\SetKwInOut{KwInput}{输入}
\SetKwInOut{KwInit}{初始化}
\KwInput{事件 $e$}
\KwInit{总接收事件数 $num=0$}
\uIf{$e$ \textnormal{属于第一种情况}}{
    $num = num + 1$\;
    $num = num - 1$\;
    $num = num + 1$\;
}
\uElseIf{$e$ \textnormal{属于第二种情况}}{
    $num = num + 1$\;
}
\Else(\tcc*[f]{其他情况}){
    \If{$num >= 0$}{
        \lIf{\textnormal{false}}{
            思考为什么这句话会被执行
        }
        $num = num + 1$\;
    }
}
\end{algorithm}

\begin{figure}[tb]
\centering
\begingroup
\definecolor{comment}{rgb}{0, 0.5, 0}
\definecolor{type}{rgb}{0.17, 0.57, 0.69}
\definecolor{number}{rgb}{0, 0.5, 0}
\definecolor{string}{rgb}{0.75, 0.08, 0.08}
\definecolor{func}{rgb}{0.54, 0.17, 0.89}
\lstset{
    backgroundcolor=\color{white},   % 背景颜色
    frame=single,                    % 边框样式
    rulecolor=\color{black},         % 边框颜色 
    basicstyle=\ttfamily\small,      % 基本字体样式
    numbers=left,                    % 行号位置
    numberstyle=\tiny\color{gray},   % 行号样式
    breaklines=true,                 % 自动换行
    commentstyle=\color{comment},    % 注释样式
    comment=[l]{//},
    stringstyle=\color{comment},
    string=[b]",
    showstringspaces=false,
    keywordstyle=[1]\color{blue},    % 关键字样式
    keywordstyle=[2]\color{type},
    keywordstyle=[3]\color{func},
    keywords=[1]{thread, temp},
    morekeywords=[2]{Client},
    morekeywords=[3]{Send, Receive, Assert, Exit},
}
\begin{lstlisting}
thread(Client) LockOp {
Acquire:
  Send(acquire_request);
End:
  temp pkt = Receive();
  Assert(pkt.type == ACK_OF_ACQUIRE,
      "unexpected packet type");
  Exit();
}
\end{lstlisting}
\endgroup
\caption{代码块示例。}
\label{fig:code-block}
\end{figure}

\section{后续章节}

后续章节为原作者对模板的使用说明,由于本人修改过模板,部分细节已不再相同。为尊重原作者的劳动成果,我没有对后续章节进行修改,阅读时请对照 \texttt{README.md} 中的修改内容说明。